\chapter{Arhitektura i dizajn sustava}

		
		\textbf{\textit{dio 1. revizije}}\\

		\textit{ Potrebno je opisati stil arhitekture te identificirati: podsustave, preslikavanje na radnu platformu, spremišta podataka, mrežne protokole, globalni upravljački tok i sklopovsko-programske zahtjeve. Po točkama razraditi i popratiti odgovarajućim skicama:}
	\begin{itemize}
		\item 	\textit{izbor arhitekture temeljem principa oblikovanja pokazanih na predavanjima (objasniti zašto ste baš odabrali takvu arhitekturu)}
		\item 	\textit{organizaciju sustava s najviše razine apstrakcije (npr. klijent-poslužitelj, baza podataka, datotečni sustav, grafičko sučelje)}
		\item 	\textit{organizaciju aplikacije (npr. slojevi frontend i backend, MVC arhitektura) }		
	\end{itemize}

	
		

		\eject

				
		\section{Baza podataka}
			
			
		Za potrebe našeg sustava koristit ćemo SQL relacijsku bazu podataka. Relacijska baza podataka najviše nam odgovara zbog olakšanog modeliranja događaja i entiteta iz stvarnog svijeta. Osnovna jedinka baze podataka je relacija, to jest tablica koja ima svoj naziv i potreban skup atributa. Zadaća baze podataka je brza i jednostavna pohrana, izmjena i dohvat podataka koje potom treba obraditi. Sve su relacije u bazi svedene na treću normalnu formu kako baza ne bi sadržavala redundantne podatke. Prilikom izrade baze podataka poslužili smo se PostgreSQL-om.
		Baza podataka ove aplikacije sastoji se od sljedećih entiteta:
		\begin{itemize}
			\item 	Profil
			\item 	Korisnički račun
			\item 	Registracija klijenta
			\item   Razina ovlasti
			\item   Kredit
			\item   Vrsta kredita
			\item   Transakcija
			\item   Račun
			\item   Vrsta računa
			\item   Kartica
			\item   Vrsta kartice	
			\item   Mjesto
			\item   Županija	
		\end{itemize}
		
		\eject
		
			\subsection{Opis tablica}
			

				\textbf{Profil} Ovaj entitet sadrži sve važne informacije za pristup web aplikaciji. Sadrži atribute: ime, prezime, OIB, adresa prebivališta, poštanski broj, datum rođenja, e-mail adresa i slika profila. Ovaj entitet u vezi je \textit{Zero-to-Many} s entitetom Korisnički račun preko OIB-a i u vezi je \textit{Many-to-One} sa entitetom Mjesto preko poštanskog broja. Također je i u vezi sa \textit{One-to-One} sa entitetom Registracija klijenta preko OIB-a.
				
				\begin{longtabu} to \textwidth {|X[8, l]|X[8, l]|X[16, l]|}
					
					\hline \multicolumn{3}{|c|}{\textbf{Profil }}	 \\[3pt] \hline
					\endfirsthead
					
					\hline \multicolumn{3}{|c|}{\textbf{Profil}}	 \\[3pt] \hline
					\endhead
					
					\hline 
					\endlastfoot
					
					Ime & VARCHAR	&  	ime korisnika\cellcolor{LightGreen} 	\\ \hline
					Prezime	& VARCHAR &  prezime korisnika 	\\ \hline 
					\rowcolor{lightgreen}OIB & VARCHAR &  OIB korisnika \\ \hline 
					Adresa prebivališta & VARCHAR &   adresa korisnika      \\ \hline
					Poštanski broj & INT & poštanski broj (mjesto.poštanskiBroj) \\ \hline
					Datum rođenja & DATE & datum rođenja korisnika \\ \hline
					Email & VARCHAR & e-mail adresa korisnika \\ \hline
					Slika & VARCHAR & slika korisnika \\ \hline
					
					 
					
					
				\end{longtabu}
			
			\textbf{Korisnički Račun}  Ovaj entitet sadrži sve važne informacije o korisničkom računu. Sadrži atribute: OIB, korisničko ime, lozinka, razina ovlasti i promjena lozinke. Ovaj entitet je u vezi \textit{Many-to-Zero} s entitetom Profil preko OIB-a i s entitetom \textit{{One-to-One}} Razina ovlasti preko razine ovlasti. 
			
			\begin{longtabu} to \textwidth {|X[8, l]|X[8, l]|X[16, l]|}
				
				\hline \multicolumn{3}{|c|}{\textbf{Korisnički račun }}	 \\[3pt] \hline
				\endfirsthead
				
				\hline \multicolumn{3}{|c|}{\textbf{Korisnički račun}}	 \\[3pt] \hline
				\endhead
				
				\hline 
				\endlastfoot
				
				\cellcolor{LightGreen}Korisničko ime & VARCHAR	&  jedinstveni identifikator korisnika \\ \hline
				Lozinka	& VARCHAR &   hash lozinke	\\ \hline 
				OIB & VARCHAR & OIB korisnika (profil.OIB)\\ \hline
				Razina ovlasti & INT & broj razine ovlasti (razinaOvlasti.razinaOvlasti)\\ \hline
				Promjena lozinske & BOOLEAN & treba li promijeniti lozinku prilikom prijave \\ \hline
				
			\end{longtabu}
		

		\eject
		
				\textbf{Registracija klijenta}  Ovaj entitet sadrži sve važne informacije o registraciji klijenta. Sadrži atribute: OIB i privremeni ključ. Ovaj entitet je u vezi \textit{One-to-One} s entitetom Profil preko OIB-a.
			
			\begin{longtabu} to \textwidth {|X[8, l]|X[8, l]|X[16, l]|}
				
				\hline \multicolumn{3}{|c|}{\textbf{Registracija klijenta }}	 \\[3pt] \hline
				\endfirsthead
				
				\hline \multicolumn{3}{|c|}{\textbf{Registracija klijenta}}	 \\[3pt] \hline
				\endhead
				
				\hline 
				\endlastfoot
				
				OIB & VARCHAR & OIB korisnika (profil.OIB)\\ \hline
				Privremeni ključ & VARCHAR & privremeni ključ prilikom otvaranja računa \\ \hline
				
			\end{longtabu}
			
			
			
		
		
			
		\textbf{Razina ovlasti}  Ovaj entitet sadrži informacije vezane za razinu ovlasti. Sadrži atribute: razina ovlasti i naziv. Ovaj je entitet u vezi \textit{One-to-One} s entitetom Korisnički račun preko razine ovlasti.  
		
		\begin{longtabu} to \textwidth {|X[8, l]|X[8, l]|X[16, l]|}
			
			\hline \multicolumn{3}{|c|}{\textbf{Razina ovlasti}}	 \\[3pt] \hline
			\endfirsthead
			
			\hline \multicolumn{3}{|c|}{\textbf{Razina ovlasti}}	 \\[3pt] \hline
			\endhead
			
			\hline 
			\endlastfoot
			
		    Razina ovlasti & INT & Broj razine ovlasti  \\ \hline
			\cellcolor{LightGreen}Naziv & VARCHAR	& Naziv ovlasti	\\ \hline
			
			
			
			
			
		\end{longtabu}
	
		\textbf{Mjesto} Ovaj entitet sadrži sve važne informacije o mjestu. Sadrži atribute: poštanski broj, naziv mjesta i šifra županije. U vezi je \textit{One-to-Many} s entitetom Profil preko poštanskog broja i u vezi \textit{Many-to-One} s entitetom Županija preko šifre županije.
			\begin{longtabu} to \textwidth {|X[8, l]|X[8, l]|X[16, l]|}
		
		\hline \multicolumn{3}{|c|}{\textbf{Mjesto}}	 \\[3pt] \hline
		\endfirsthead
		
		\hline \multicolumn{3}{|c|}{\textbf{Mjesto}}	 \\[3pt] \hline
		\endhead
		
		\hline 
		\endlastfoot
		
		Poštanski broj & INT & poštanski broj mjesta \\ \hline
		Naziv mjesta & VARCHAR & naziv mjesta \\ \hline
		Šifra županije & INT & šifra županije(županija.šifraŽupanije) \\ \hline
		
		
		
	\end{longtabu}


		\textbf{Županija} Ovaj entitet sadrži sve važne informacije o županiji. Sadrži atribute: šifra županije i naziv županije. U vezi u vezi \textit{One-to-Many} s entitetom Mjesto preko šifre županije.
		\begin{longtabu} to \textwidth {|X[8, l]|X[8, l]|X[16, l]|}
			
			\hline \multicolumn{3}{|c|}{\textbf{Županija}}	 \\[3pt] \hline
			\endfirsthead
			
			\hline \multicolumn{3}{|c|}{\textbf{Županija}}	 \\[3pt] \hline
			\endhead
			
			\hline 
			\endlastfoot
			
			
			Šifra županije & INT & šifra županije \\ \hline
			Naziv županije & VARCHAR & naziv županije \\ \hline
			
			
			
		\end{longtabu}
			
				\eject
		
			\textbf{Kredit}   Ovaj entitet sadrži sve važne informacije o kreditu koji klijent zatraži. Sadrži atribute: broj kredita, OIB, iznos, vrsta kredita, datum ugovaranja, vremenski period otplate kredita, datum rate i preostalo dugovanje. U vezi je \textit{One-to-One} s entitetom Vrsta Kredita preko vrste kredita i u vezi \textit{Many-to-One} s entitetom Profil preko OIB-a.
			
		
			\begin{longtabu} to \textwidth {|X[8, l]|X[8, l]|X[16, l]|}
			
			\hline \multicolumn{3}{|c|}{\textbf{Kredit}}	 \\[3pt] \hline
			\endfirsthead
			
			\hline \multicolumn{3}{|c|}{\textbf{Kredit}}	 \\[3pt] \hline
			\endhead
			
			\hline 
			\endlastfoot
			
			Broj kredita & INT & broj kredita \\ \hline
			OIB & VARCHAR & OIB korisnika (profil.OIB)\\ \hline
			Iznos & DECIMAL (10, 2) & iznos kredita \\ \hline
			Vrsta & INT & vrsta kredita (vrstaKredita.tipKredita) \\ \hline
			Datum ugovaranja & DATE & datum ugovaranja kredita \\ \hline
			Vremenski period otplate & INT & duljina otplate kredita u godinama \\ \hline
			Datum rate & INT & datum plaćanja mjesečne rate \\ \hline
			Preostalo dugovanje & DECIMAL(10,2) & preostali iznos dugovanja \\ \hline
			
			
			
			
		\end{longtabu}
	
			\textbf{Vrsta kredita} Ovaj entitet sadrži sve važne informacije o vrsti kredita. Sadrži atribute: vrsta kredita, naziv vrste kredita i kamatnu stopu. U vezi je \textit{One-to-One} sa entitetom Kredit preko tipa kredita. 
	
			\begin{longtabu} to \textwidth {|X[8, l]|X[8, l]|X[16, l]|}
		
			\hline \multicolumn{3}{|c|}{\textbf{Vrsta kredita}}	 \\[3pt] \hline
			\endfirsthead
		
			\hline \multicolumn{3}{|c|}{\textbf{Vrsta kredita}}	 \\[3pt] \hline
			\endhead
		
			\hline 
			\endlastfoot
			
			Vrsta kredita & INT & broj vrste kredita\\ \hline
			Naziv vrste kredita & VARCHAR & vrsta kredita\\ \hline
			Kamatna stopa & DECIMAL (3,2) &iznos kamatne stope\\ \hline
		
		
		\end{longtabu}
	
		\eject
		
				\textbf{Transakcija}   Ovaj entitet sadrži sve važne informacije o transakcijama koje klijent želi provesti. Sadrži atribute: broj transakcije, broj računa terećenja, račun odobrenja, iznos i datum transakcije. Entitet Transakcija u vezi je \textit{Many-to-One} s entitetom Račun preko broja računa terećenja i broja računa odobrenja.
			
			\begin{longtabu} to \textwidth {|X[8, l]|X[8, l]|X[16, l]|}
				
				\hline \multicolumn{3}{|c|}{\textbf{Transakcija}}	 \\[3pt] \hline
				\endfirsthead
				
				\hline \multicolumn{3}{|c|}{\textbf{Transakcija}}	 \\[3pt] \hline
				\endhead
				
				\hline 
				\endlastfoot
				
				Broj transakcije & INT & broj transakcije\\ \hline
				Račun terećenja & VARCHAR & broj računa terećenja\\ \hline
				Račun odobrenja & VARCHAR & broj računa odobrenja\\ \hline
				Iznos & DECIMAL (10,2) & iznos uplate\\ \hline
				Datum transakcije & DATE & datum transakcije\\ \hline
			
				
				
				
			\end{longtabu}
		
			\textbf{Račun}   Ovaj entitet sadrži sve važne informacije o računu koji ima klijent. Sadrži atribute: broj računa, OIB klijenta, datum otvaranja računa, stanje računa, vrsta računa, prekoračenje, kamatna stopa i datum zatvaranja. U vezi je  \textit{Many-to-One} s entitetom Profil preko OIB-a, sa entitetom Vrsta Računa \textit{One-to-One} preko vrste računa. Također je u vezi \textit{One-to-Many} s entitetom Transakcija preko broja računa.
		
			\begin{longtabu} to \textwidth {|X[8, l]|X[8, l]|X[16, l]|}
			
			\hline \multicolumn{3}{|c|}{\textbf{Račun}}	 \\[3pt] \hline
			\endfirsthead
			
			\hline \multicolumn{3}{|c|}{\textbf{Račun}}	 \\[3pt] \hline
			\endhead
			
			\hline 
			\endlastfoot
			
			Broj računa & VARCHAR & broj računa \\ \hline
			OIB & VARCHAR & OIB korisnika (profil.OIB) \\ \hline
			Datum otvaranja & DATE & datum otvaranja računa \\ \hline
			Stanje & DECIMAL (10, 2) & trenutno stanje računa \\ \hline
			Vrsta računa & INT & broj vrste računa (vrstaRačuna.vrstaRačuna) \\ \hline
			Prekoračenje & DECIMAL(10,2) & iznos prekoračenja računa \\ \hline
			Kamatna stopa & DECIMAL(3, 2) & iznos kamatne stope \\ \hline
			Datum zatvaranja & DATE & datum zatvaranja računa \\ \hline
			
			
			
			
			
			
		\end{longtabu}
	
				\textbf{Vrsta računa}   Ovaj entitet sadrži sve važne informacije o vrsti računa koje neki klijent posjeduje. Sadrži atribute: broj vrste računa i naziv računa. U vezi je \textit{One-to-One} s entitetom Račun preko vrste računa.
			\begin{longtabu} to \textwidth {|X[8, l]|X[8, l]|X[16, l]|}
				
				\hline \multicolumn{3}{|c|}{\textbf{Vrsta računa}}	 \\[3pt] \hline
				\endfirsthead
				
				\hline \multicolumn{3}{|c|}{\textbf{Vrsta računa}}	 \\[3pt] \hline
				\endhead
				
				\hline 
				\endlastfoot
				
				Vrsta računa & INT & Broj vrste računa \\ \hline
				Naziv računa & VARCHAR & Naziv vrste računa \\ \hline
				
				
				
				
	\end{longtabu}	
		
		
			\textbf{Kartica}   Ovaj entitet sadrži sve važne informacije o karticama koje ima klijent. Sadrži atribute: broj kartice, broj računa, OIB, broj vrste kartice, stanje, valjanost,limit, kamatna stopa i datum rate. U vezi je \textit{Many-to-One} s entitetom Račun preko broja računa i sa entitetom Vrsta kartice \textit{One-to-One} preko vrste kartice. 
			
			\begin{longtabu} to \textwidth {|X[8, l]|X[8, l]|X[16, l]|}
				
				\hline \multicolumn{3}{|c|}{\textbf{Kartica}}	 \\[3pt] \hline
				\endfirsthead
				
				\hline \multicolumn{3}{|c|}{\textbf{Kartica}}	 \\[3pt] \hline
				\endhead
				
				\hline 
				\endlastfoot
				
				Broj kartice & VARCHAR & broj kartice\\ \hline
				Broj računa & VARCHAR & broj računa za koji je kartica vezana (račun.brojRačuna) \\ \hline
				OIB & VARCHAR & OIB klijenta (profil.OIB)\\ \hline
				Vrsta kartice & INT & broj tipa kartice (vrstaKartice.tipKartice)\\ \hline
				Stanje & DECIMAL(10,2) & stanje kartice\\ \hline
				Valjanost & DATE & datum isteka valjanosti kartice \\ \hline
				Limit & DECIMAL (10,2)& odobren limit \\ \hline
				Kamatna stopa & DECIMAL(3, 2) & iznos kamatne stope \\ \hline
				Datum rate & INT & datum otplate dugovanja kartice \\ \hline
	
		

		\end{longtabu}	
	
			\textbf{Vrsta kartice}   Ovaj entitet sadrži sve važne informacije o vrstama kartice koje ima klijent. Sadrži atribute:broj vrste kartice i naziv kartice. U vezi je \textit{One-to-One} s entitetom Kartica preko vrste kartice.
				\begin{longtabu} to \textwidth {|X[8, l]|X[8, l]|X[16, l]|}
				
				\hline \multicolumn{3}{|c|}{\textbf{Vrsta kartice}}	 \\[3pt] \hline
				\endfirsthead
				
				\hline \multicolumn{3}{|c|}{\textbf{Vrsta kartice}}	 \\[3pt] \hline
				\endhead
				
				\hline 
				\endlastfoot
		
				Vrsta kartice & INT & broj vrste kartice\\ \hline
				Naziv kartice & VARCHAR & naziv kartice\\ \hline
		
		
		
		\end{longtabu}
	
				
		
		
		
			
			\subsection{Dijagram baze podataka}
				\begin{figure}[H]
					\includegraphics[scale=0.8]{Slike/ermodel.PNG}
					\centering
					\caption{E-R dijagram baze podataka}
					\label{fig:dijagram}
				\end{figure}
			\eject
			
			
		\section{Dijagram razreda}
		
			\textit{Potrebno je priložiti dijagram razreda s pripadajućim opisom. Zbog preglednosti je moguće dijagram razlomiti na više njih, ali moraju biti grupirani prema sličnim razinama apstrakcije i srodnim funkcionalnostima.}\\
			
			\textbf{\textit{dio 1. revizije}}\\
			
			\textit{Prilikom prve predaje projekta, potrebno je priložiti potpuno razrađen dijagram razreda vezan uz \textbf{generičku funkcionalnost} sustava. Ostale funkcionalnosti trebaju biti idejno razrađene u dijagramu sa sljedećim komponentama: nazivi razreda, nazivi metoda i vrste pristupa metodama (npr. javni, zaštićeni), nazivi atributa razreda, veze i odnosi između razreda.}\\
			
			\textbf{\textit{dio 2. revizije}}\\			
			
			\textit{Prilikom druge predaje projekta dijagram razreda i opisi moraju odgovarati stvarnom stanju implementacije}
			
			
			
			\eject
		
		\section{Dijagram stanja}
			
			
			\textbf{\textit{dio 2. revizije}}\\
			
			\textit{Potrebno je priložiti dijagram stanja i opisati ga. Dovoljan je jedan dijagram stanja koji prikazuje \textbf{značajan dio funkcionalnosti} sustava. Na primjer, stanja korisničkog sučelja i tijek korištenja neke ključne funkcionalnosti jesu značajan dio sustava, a registracija i prijava nisu. }
			
			
			\eject 
		
		\section{Dijagram aktivnosti}
			
			\textbf{\textit{dio 2. revizije}}\\
			
			 \textit{Potrebno je priložiti dijagram aktivnosti s pripadajućim opisom. Dijagram aktivnosti treba prikazivati značajan dio sustava.}
			
			\eject
		\section{Dijagram komponenti}
		
			\textbf{\textit{dio 2. revizije}}\\
		
			 \textit{Potrebno je priložiti dijagram komponenti s pripadajućim opisom. Dijagram komponenti treba prikazivati strukturu cijele aplikacije.}