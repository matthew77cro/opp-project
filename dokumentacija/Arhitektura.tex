\chapter{Arhitektura i dizajn sustava}

	Arhitektura aplikacije može se podijeliti na četiri podsustava:
	\begin{itemize}
		\item Web poslužitelj
		\item Web aplikacija
		\item Baza podataka
		\item Mobilna aplikacija
	\end{itemize}

	\begin{figure}[H]
		\includegraphics[scale=0.7]{Slike/Arhitektura sustava.PNG}
		\centering
		\caption{Arhitektura sustava s prikazom podsustava i toka podataka}
		\label{fig:dijagramSustava}
	\end{figure}

	\textit{\underline{Web poslužitelj}} osnova je web aplikacije. To je računalo koje preko internetske mreže uspostavlja komunikaciju s korisnicima sustava. U našem slučaju, to je Windows Server virtualna mašina podignuta preko servisa Microsoft Azure. Na njemu su instalirani svi servisi koji su potpora radu web aplikacije i baze podataka.
	
	\textit{\underline{Web aplikacija}} mozak je našeg sustava. U njoj je opisana sva logika aplikacije. Razgovara sa korisnikom sustava putem HTTP (engl. \textit{Hyper
	Text Transfer Protocol}) protokola, obrađuje zahtjeve, spaja se s bazom podataka te šalje odgovor korisniku na web preglednik ili Android aplikaciju. Odgovor je oblika HTML dokumenta u slučaju korištenja web preglednika ili oblika JSON (JavaScript Object Notation) dokumenta u slučaju spajanja Android aplikacije na RESTful API sustava.

	Za backend sustava odabran je programski jezik Java te tehnologije Java Servleti i JSP (Java Server Pages). Backend je pisan kao Maven projekt radi lakšeg upravljanja ovisnostima naše aplikacije. Kao razvojno okruženje koristi se IDE Eclipse, a Apache Tomcat kao Servlet container.
	
	Oblikovnim obrascem MVC (Model-View-Controller) ostvarili smo razdvajanje frontend i backend dijela sustava te njihovo nezavisno i paralelno razvijanje. Posljedica toga jest jednostavno testiranje i utvrđivanje grešaka sustava te jednostavna buduća nadogradnja.
	
	Obrazac MVC sastoji se od tri komponente:
	\begin{packed_item}
		
		\item \textbf{Model} - strukture podataka koje služe preslikavanju podataka iz baze u objektno orijentiran jezik. Omogućuju lakši dohvat i lakše upravljanje podacima spremljenim u bazu.
		\item \textbf{View} - komponenta koja služi prikazu podataka iz modela na korisniku razumljiv i jednostavan način. U slučaju naše web aplikacije, za to je zadužena tehnologija JSP.
		\item \textbf{Controller} - komponenta koja služi upravljanju (sprema i dohvaća podataka iz baze, izvršava operacija nad podacima i cijelim sustavom, priprema podatke za prikaz i dr.)
		
	\end{packed_item}
	
	Odabrali smo relacijski model \textit{\underline{baze podataka}} zbog njegove raširenosti, jednostavnosti i skalabilnosti. Bazu podataka opisali smo jezikom SQL i pokrenuli ju na sustavu PostgreSQL.

	\textit{\underline{Mobilna aplikacija}} još jedan je pogled korisnika u naš sustav. Ona je izgrađena u programskom jeziku Java koristeći Android studio kao IDE. Mobilna aplikacija spaja se na RESTful dio web poslužitelja preko kojeg vrši prijavu u sustav i dohvaćanje podataka. Za izradu dizajna Android aplikacije koristi se sustav Adobe XD i Google Material Design kao vizualni jezik.
		
		\eject
				
		\section{Baza podataka}
			
			
		Za potrebe našeg sustava koristit ćemo SQL relacijsku bazu podataka. Relacijska baza podataka najviše nam odgovara zbog olakšanog modeliranja događaja i entiteta iz stvarnog svijeta. Osnovna jedinka baze podataka je relacija, to jest tablica koja ima svoj naziv i potreban skup atributa. Zadaća baze podataka je brza i jednostavna pohrana, izmjena i dohvat podataka koje potom treba obraditi. Sve su relacije u bazi svedene na treću normalnu formu kako baza ne bi sadržavala redundantne podatke. Prilikom izrade baze podataka poslužili smo se PostgreSQL-om.
		Baza podataka ove aplikacije sastoji se od sljedećih entiteta:
		\begin{itemize}
			\item 	Profil
			\item 	Korisnički račun
			\item 	Registracija klijenta
			\item   Razina ovlasti
			\item   Kredit
			\item   Vrsta kredita
			\item  	Zahtjev kredit
			\item   Transakcija
			\item   Račun
			\item   Vrsta računa
			\item   Kartica
			\item   Vrsta kartice
			\item 	Zahtjev kartica	
			\item   Mjesto
			\item   Županija	
		\end{itemize}
		
		\eject
		
			\subsection{Opis tablica}
			

				\textbf{Profil} Ovaj entitet sadrži sve važne informacije za pristup web aplikaciji. Sadrži atribute: ime, prezime, OIB, adresa prebivališta, poštanski broj, datum rođenja, e-mail adresa i slika profila. Ovaj entitet u vezi je \textit{Zero-to-Many} s entitetom Korisnički račun preko OIB-a i u vezi je \textit{Many-to-One} sa entitetom Mjesto preko poštanskog broja. Također je i u vezi sa \textit{One-to-One} sa entitetom Registracija klijenta preko OIB-a.
				
				\begin{longtabu} to \textwidth {|X[8, l]|X[8, l]|X[16, l]|}
					
					\hline \multicolumn{3}{|c|}{\textbf{Profil }}	 \\[3pt] \hline
					\endfirsthead
					
					\hline \multicolumn{3}{|c|}{\textbf{Profil}}	 \\[3pt] \hline
					\endhead
					
					\hline 
					\endlastfoot
					
					Ime & VARCHAR	&  	ime korisnika\cellcolor{LightGreen} 	\\ \hline
					Prezime	& VARCHAR &  prezime korisnika 	\\ \hline 
					\rowcolor{lightgreen}\textbf{OIB} & VARCHAR &  OIB korisnika \\ \hline 
					Adresa prebivališta & VARCHAR &   adresa korisnika      \\ \hline
					\textit{Poštanski broj} & INT & poštanski broj (mjesto.poštanskiBroj) \\ \hline
					Datum rođenja & DATE & datum rođenja korisnika \\ \hline
					Email & VARCHAR & e-mail adresa korisnika \\ \hline
					Slika & VARCHAR & slika korisnika \\ \hline
					
					 
					
					
				\end{longtabu}
			
			\textbf{Korisnički Račun}  Ovaj entitet sadrži sve važne informacije o korisničkom računu. Sadrži atribute: OIB, korisničko ime, lozinka, razina ovlasti i promjena lozinke. Ovaj entitet je u vezi \textit{Many-to-Zero} s entitetom Profil preko OIB-a i s entitetom \textit{{One-to-One}} Razina ovlasti preko razine ovlasti. 
			
			\begin{longtabu} to \textwidth {|X[8, l]|X[8, l]|X[16, l]|}
				
				\hline \multicolumn{3}{|c|}{\textbf{Korisnički račun }}	 \\[3pt] \hline
				\endfirsthead
				
				\hline \multicolumn{3}{|c|}{\textbf{Korisnički račun}}	 \\[3pt] \hline
				\endhead
				
				\hline 
				\endlastfoot
				
				\cellcolor{LightGreen}\textbf{Korisničko ime} & VARCHAR	&  jedinstveni identifikator korisnika \\ \hline
				Lozinka	& VARCHAR &   hash lozinke	\\ \hline 
				\textit{OIB} & VARCHAR & OIB korisnika (profil.OIB)\\ \hline
				\textit{Razina ovlasti} & INT & broj razine ovlasti (razinaOvlasti.razinaOvlasti)\\ \hline
				Promjena lozinske & BOOLEAN & treba li promijeniti lozinku prilikom prijave \\ \hline
				
			\end{longtabu}
		

		\eject
		
				\textbf{Registracija klijenta}  Ovaj entitet sadrži sve važne informacije o registraciji klijenta. Sadrži atribute: OIB i privremeni ključ. Ovaj entitet je u vezi \textit{One-to-One} s entitetom Profil preko OIB-a.
			
			\begin{longtabu} to \textwidth {|X[8, l]|X[8, l]|X[16, l]|}
				
				\hline \multicolumn{3}{|c|}{\textbf{Registracija klijenta }}	 \\[3pt] \hline
				\endfirsthead
				
				\hline \multicolumn{3}{|c|}{\textbf{Registracija klijenta}}	 \\[3pt] \hline
				\endhead
				
				\hline 
				\endlastfoot
				
				\textit{OIB} & VARCHAR & OIB korisnika (profil.OIB)\\ \hline
				Privremeni ključ & VARCHAR & privremeni ključ prilikom otvaranja računa \\ \hline
				
			\end{longtabu}
			
			
			
		
		
			
		\textbf{Razina ovlasti}  Ovaj entitet sadrži informacije vezane za razinu ovlasti. Sadrži atribute: razina ovlasti i naziv. Ovaj je entitet u vezi \textit{One-to-One} s entitetom Korisnički račun preko razine ovlasti.  
		
		\begin{longtabu} to \textwidth {|X[8, l]|X[8, l]|X[16, l]|}
			
			\hline \multicolumn{3}{|c|}{\textbf{Razina ovlasti}}	 \\[3pt] \hline
			\endfirsthead
			
			\hline \multicolumn{3}{|c|}{\textbf{Razina ovlasti}}	 \\[3pt] \hline
			\endhead
			
			\hline 
			\endlastfoot
			
		    \textbf{Razina ovlasti} & INT & Broj razine ovlasti  \\ \hline
			Naziv & VARCHAR	& Naziv ovlasti	\\ \hline
			
			
			
			
			
		\end{longtabu}
	
		\textbf{Mjesto} Ovaj entitet sadrži sve važne informacije o mjestu. Sadrži atribute: poštanski broj, naziv mjesta i šifra županije. U vezi je \textit{One-to-Many} s entitetom Profil preko poštanskog broja i u vezi \textit{Many-to-One} s entitetom Županija preko šifre županije.
			\begin{longtabu} to \textwidth {|X[8, l]|X[8, l]|X[16, l]|}
		
		\hline \multicolumn{3}{|c|}{\textbf{Mjesto}}	 \\[3pt] \hline
		\endfirsthead
		
		\hline \multicolumn{3}{|c|}{\textbf{Mjesto}}	 \\[3pt] \hline
		\endhead
		
		\hline 
		\endlastfoot
		
		\textbf{Poštanski broj} & INT & poštanski broj mjesta \\ \hline
		Naziv mjesta & VARCHAR & naziv mjesta \\ \hline
		\textit{Šifra županije} & INT & šifra županije(županija.šifraŽupanije) \\ \hline
		
		
		
	\end{longtabu}


		\textbf{Županija} Ovaj entitet sadrži sve važne informacije o županiji. Sadrži atribute: šifra županije i naziv županije. U vezi u vezi \textit{One-to-Many} s entitetom Mjesto preko šifre županije.
		\begin{longtabu} to \textwidth {|X[8, l]|X[8, l]|X[16, l]|}
			
			\hline \multicolumn{3}{|c|}{\textbf{Županija}}	 \\[3pt] \hline
			\endfirsthead
			
			\hline \multicolumn{3}{|c|}{\textbf{Županija}}	 \\[3pt] \hline
			\endhead
			
			\hline 
			\endlastfoot
			
			
			\textbf{Šifra županije} & INT & šifra županije \\ \hline
			Naziv županije & VARCHAR & naziv županije \\ \hline
			
			
			
		\end{longtabu}
			
				\eject
		
			\textbf{Kredit}   Ovaj entitet sadrži sve važne informacije o kreditu koji klijent zatraži. Sadrži atribute: broj kredita, OIB, iznos, vrsta kredita, datum ugovaranja, vremenski period otplate kredita, datum rate i preostalo dugovanje. U vezi je \textit{One-to-One} s entitetom Vrsta Kredita preko vrste kredita i u vezi \textit{Many-to-One} s entitetom Profil preko OIB-a.
			
		
			\begin{longtabu} to \textwidth {|X[8, l]|X[8, l]|X[16, l]|}
			
			\hline \multicolumn{3}{|c|}{\textbf{Kredit}}	 \\[3pt] \hline
			\endfirsthead
			
			\hline \multicolumn{3}{|c|}{\textbf{Kredit}}	 \\[3pt] \hline
			\endhead
			
			\hline 
			\endlastfoot
			
			\textbf{Broj kredita} & INT & broj kredita \\ \hline
			\textit{OIB} & VARCHAR & OIB korisnika (profil.OIB)\\ \hline
			Iznos & DECIMAL (10, 2) & iznos kredita \\ \hline
			\textit{Vrsta} & INT & vrsta kredita (vrstaKredita.tipKredita) \\ \hline
			Datum ugovaranja & DATE & datum ugovaranja kredita \\ \hline
			Vremenski period otplate & INT & duljina otplate kredita u godinama \\ \hline
			Datum rate & INT & datum plaćanja mjesečne rate \\ \hline
			Preostalo dugovanje & DECIMAL(10,2) & preostali iznos dugovanja \\ \hline
			
			
			
			
		\end{longtabu}
	
			\textbf{Vrsta kredita} Ovaj entitet sadrži sve važne informacije o vrsti kredita. Sadrži atribute: vrsta kredita, naziv vrste kredita i kamatnu stopu. U vezi je \textit{One-to-One} sa entitetom Kredit preko vrste kredita i u vezi {One-to-Many} sa entitetom Zahtjev kredita preko vrste kredita. 
	
			\begin{longtabu} to \textwidth {|X[8, l]|X[8, l]|X[16, l]|}
		
			\hline \multicolumn{3}{|c|}{\textbf{Vrsta kredita}}	 \\[3pt] \hline
			\endfirsthead
		
			\hline \multicolumn{3}{|c|}{\textbf{Vrsta kredita}}	 \\[3pt] \hline
			\endhead
		
			\hline 
			\endlastfoot
			
			\textbf{Vrsta kredita} & INT & broj vrste kredita\\ \hline
			Naziv vrste kredita & VARCHAR & vrsta kredita\\ \hline
			Kamatna stopa & DECIMAL (3,2) &iznos kamatne stope\\ \hline
		
		
		\end{longtabu}
	
		\eject
		
			\textbf{Zahtjev kredita}   Ovaj entitet sadrži sve važne informacije o zahtjevu kredita. Sadrži atribute: šifra zahtjeva, OIB, iznos, šifra vrste kredita, period otplate i odobrenje. U vezi je \textit{One-to-One} s entitetom Vrsta Kredita preko šifre vrste kredita i u vezi \textit{Many-to-One} s entitetom Profil preko OIB-a.
		
		
		\begin{longtabu} to \textwidth {|X[8, l]|X[8, l]|X[16, l]|}
			
			\hline \multicolumn{3}{|c|}{\textbf{Zahtjev kredita}}	 \\[3pt] \hline
			\endfirsthead
			
			\hline \multicolumn{3}{|c|}{\textbf{Zahtjev kredita}}	 \\[3pt] \hline
			\endhead
			
			\hline 
			\endlastfoot
			
			\textbf{Šifra zahtjeva} & INT & šifra zahtjeva kredita \\ \hline
			\textit{OIB} & VARCHAR & OIB korisnika (profil.OIB)\\ \hline
			Iznos & DECIMAL (10, 2) & iznos kredita \\ \hline
			\textit{Šifra vrste kredita} & INT & vrsta kredita (šifraVrsteKredita.tipKredita) \\ \hline
			Period otplate & INT & duljina otplate kredita u godinama \\ \hline
			Odobrenje & BOOLEAN & status zahtjeva kredita \\ \hline
			
			
			
			
		\end{longtabu}
		
				\textbf{Transakcija}   Ovaj entitet sadrži sve važne informacije o transakcijama koje klijent želi provesti. Sadrži atribute: broj transakcije, broj računa terećenja, račun odobrenja, iznos i datum transakcije. Entitet Transakcija u vezi je \textit{Many-to-One} s entitetom Račun preko broja računa terećenja i broja računa odobrenja.
			
			\begin{longtabu} to \textwidth {|X[8, l]|X[8, l]|X[16, l]|}
				
				\hline \multicolumn{3}{|c|}{\textbf{Transakcija}}	 \\[3pt] \hline
				\endfirsthead
				
				\hline \multicolumn{3}{|c|}{\textbf{Transakcija}}	 \\[3pt] \hline
				\endhead
				
				\hline 
				\endlastfoot
				
				\textbf{Broj transakcije} & INT & broj transakcije\\ \hline
				\textit{Račun terećenja} & VARCHAR & broj računa terećenja(račun.brojRačuna)\\ \hline
				\textit{Račun odobrenja} & VARCHAR & broj računa odobrenja(račun.brojRačuna)\\ \hline
				Iznos & DECIMAL (10,2) & iznos uplate\\ \hline
				Datum transakcije & DATE & datum transakcije\\ \hline
			
				
				
				
			\end{longtabu}
		
		
			\textbf{Račun}   Ovaj entitet sadrži sve važne informacije o računu koji ima klijent. Sadrži atribute: broj računa, OIB klijenta, datum otvaranja računa, stanje računa, vrsta računa, prekoračenje, kamatna stopa i datum zatvaranja. U vezi je  \textit{Many-to-One} s entitetom Profil preko OIB-a, sa entitetom Vrsta Računa \textit{One-to-One} preko vrste računa. Također je u vezi \textit{One-to-Many} s entitetom Transakcija preko broja računa.
		
			\begin{longtabu} to \textwidth {|X[8, l]|X[8, l]|X[16, l]|}
			
			\hline \multicolumn{3}{|c|}{\textbf{Račun}}	 \\[3pt] \hline
			\endfirsthead
			
			\hline \multicolumn{3}{|c|}{\textbf{Račun}}	 \\[3pt] \hline
			\endhead
			
			\hline 
			\endlastfoot
			
			\textbf{Broj računa} & VARCHAR & broj računa \\ \hline
			\textit{OIB} & VARCHAR & OIB korisnika (profil.OIB) \\ \hline
			Datum otvaranja & DATE & datum otvaranja računa \\ \hline
			Stanje & DECIMAL (10, 2) & trenutno stanje računa \\ \hline
			\textit{Vrsta računa} & INT & broj vrste računa (vrstaRačuna.vrstaRačuna) \\ \hline
			Prekoračenje & DECIMAL(10,2) & iznos prekoračenja računa \\ \hline
			Kamatna stopa & DECIMAL(3, 2) & iznos kamatne stope \\ \hline
			Datum zatvaranja & DATE & datum zatvaranja računa \\ \hline		
			
		\end{longtabu}
	
			
	
				\textbf{Vrsta računa}   Ovaj entitet sadrži sve važne informacije o vrsti računa koje neki klijent posjeduje. Sadrži atribute: broj vrste računa i naziv računa. U vezi je \textit{One-to-One} s entitetom Račun preko vrste računa.
			\begin{longtabu} to \textwidth {|X[8, l]|X[8, l]|X[16, l]|}
				
				\hline \multicolumn{3}{|c|}{\textbf{Vrsta računa}}	 \\[3pt] \hline
				\endfirsthead
				
				\hline \multicolumn{3}{|c|}{\textbf{Vrsta računa}}	 \\[3pt] \hline
				\endhead
				
				\hline 
				\endlastfoot
				
				\textbf{Vrsta računa} & INT & Broj vrste računa \\ \hline
				Naziv računa & VARCHAR & Naziv vrste računa \\ \hline
				
				
				
				
	\end{longtabu}	
		
		
			\textbf{Kartica}   Ovaj entitet sadrži sve važne informacije o karticama koje ima klijent. Sadrži atribute: broj kartice, broj računa, OIB, broj vrste kartice, stanje, valjanost,limit, kamatna stopa i datum rate. U vezi je \textit{Many-to-One} s entitetom Račun preko broja računa i sa entitetom Vrsta kartice \textit{One-to-One} preko vrste kartice. 
			
			\begin{longtabu} to \textwidth {|X[8, l]|X[8, l]|X[16, l]|}
				
				\hline \multicolumn{3}{|c|}{\textbf{Kartica}}	 \\[3pt] \hline
				\endfirsthead
				
				\hline \multicolumn{3}{|c|}{\textbf{Kartica}}	 \\[3pt] \hline
				\endhead
				
				\hline 
				\endlastfoot
				
				\textbf{Broj kartice} & VARCHAR & broj kartice\\ \hline
				\textit{Broj računa} & VARCHAR & broj računa za koji je kartica vezana (račun.brojRačuna) \\ \hline
				\textit{OIB} & VARCHAR & OIB klijenta (profil.OIB)\\ \hline
				\textit{Vrsta kartice} & INT & broj tipa kartice (vrstaKartice.tipKartice)\\ \hline
				Stanje & DECIMAL(10,2) & stanje kartice\\ \hline
				Valjanost & DATE & datum isteka valjanosti kartice \\ \hline
				Limit & DECIMAL (10,2)& odobren limit \\ \hline
				Kamatna stopa & DECIMAL(3, 2) & iznos kamatne stope \\ \hline
				Datum rate & INT & datum otplate dugovanja kartice \\ \hline
	
		

		\end{longtabu}
	
	\eject	
	
			\textbf{Vrsta kartice}   Ovaj entitet sadrži sve važne informacije o vrstama kartice koje ima klijent. Sadrži atribute: broj vrste kartice i naziv kartice. U vezi je \textit{One-to-One} s entitetom Kartica preko vrste kartice i u vezi sa entitetom Zahtjev kartice {One-to-One} preko vrste kartice.
				\begin{longtabu} to \textwidth {|X[8, l]|X[8, l]|X[16, l]|}
				
				\hline \multicolumn{3}{|c|}{\textbf{Vrsta kartice}}	 \\[3pt] \hline
				\endfirsthead
				
				\hline \multicolumn{3}{|c|}{\textbf{Vrsta kartice}}	 \\[3pt] \hline
				\endhead
				
				\hline 
				\endlastfoot
		
				\textbf{Vrsta kartice} & INT & broj vrste kartice\\ \hline
				Naziv kartice & VARCHAR & naziv kartice\\ \hline
		
		
		
		\end{longtabu}
	
	
	
			\textbf{Zahtjev kartice}   Ovaj entitet sadrži sve važne informacije o zahtjevu kartice. Sadrži atribute: šifra zahtjeva, OIB, šifra vrste kartice i odobrenje. U vezi je \textit{Many-to-One} s entitetom Profil preko OIB-a i sa entitetom Vrsta kartice \textit{One-to-One} preko šifre vrste kartice. 
			
			\begin{longtabu} to \textwidth {|X[8, l]|X[8, l]|X[16, l]|}
				
				\hline \multicolumn{3}{|c|}{\textbf{Zahtjev kartice}}	 \\[3pt] \hline
				\endfirsthead
				
				\hline \multicolumn{3}{|c|}{\textbf{Zahtjev kartice}}	 \\[3pt] \hline
				\endhead
				
				\hline 
				\endlastfoot
				
				\textbf{Šifra zahtjeva} & INT & šifra zahtjeva kartice\\ \hline
				\textit{OIB} & VARCHAR & OIB klijenta (profil.OIB)\\ \hline
				\textit{Šifra vrste kartice} & INT & broj tipa kartice (vrstaKartice.tipKartice)\\ \hline
				Odobrenje & BOOLEAN & status zahtjeva kartice \\ \hline
				
				
				
			\end{longtabu}	
			
			
			\subsection{Dijagram baze podataka}
			
				\begin{figure}[H]
					\includegraphics[scale=0.9]{Slike/ermodel.PNG}
					\centering
					\caption{E-R dijagram baze podataka}
					\label{fig:dijagram}
				\end{figure}
			\eject
			
			
		\section{Dijagram razreda}
		
		Na slikama 4.3 - 4.14 prikazani su razredi koji pripadaju backend dijelu MVC
		arhitekture.
		
			\begin{figure}[H]
				\includegraphics[scale=0.54]{Slike/Class Diagram0.PNG}
				\centering
				\caption{Razredi modela opisa korisnika}
				\label{fig:dijagram}
			\end{figure}
		
			\begin{figure}[H]
				\includegraphics[scale=0.45]{Slike/Class Diagram1.PNG}
				\centering
				\caption{Ostali razredi modela}
				\label{fig:dijagram}
			\end{figure}
		
			\begin{figure}[H]
				\includegraphics[scale=0.45]{Slike/Class Diagram2.PNG}
				\centering
				\caption{Ostali razredi modela - nastavak}
				\label{fig:dijagram}
			\end{figure}
		
		Slike 4.3, 4.4 i 4.5 prikazuju strukture podataka pomoću kojih su oblikovani i prikazani podaci dohvaćeni iz baze podataka. Predstavljaju jedan-na-jedan presliku relacija baze podataka.
		
			\begin{figure}[H]
				\includegraphics[scale=0.6]{Slike/Class Diagram3.PNG}
				\centering
				\caption{Razredi zaduženi za kontrolu autentifikacije}
				\label{fig:dijagram}
			\end{figure}
		
		SLika 4.6 prikazuje razrede zadužene za kontrolu autentifikacije korisnika sustava. Razred LoginHandler preko DAO sloja komunicira s bazom podataka i uspoređuje vjerodajnice za prijavu i registraciju s onima spremljenim u bazu. Tu su i funkcionalnosti koje se koriste na više mjesta kroz aplikaciju objedinjene u jedan razred - Util.
		
			\begin{figure}[H]
				\includegraphics[scale=0.65]{Slike/Class Diagram4.PNG}
				\centering
				\caption{Razredi zaduženi za inicijalizaciju aplikacije i konekcije}
				\label{fig:dijagram}
			\end{figure}
		
		Metode razreda ConnectionSetterFilter izvode se prilikom svakog HTTP zahtjeva upućenog od strane korisnika sustava. Služi za postavljanje konekcije prema bazi podataka trenutnoj dretvi te za vraćanje konekcije u bazen konekcija po završetku obrade zahtjeva. \newline
		Metode razreda Initialisation izvode prilikom pokretanja web aplikacije. One inicijaliziraju bazen konekcija prema bazi podataka te (ukoliko je potrebno) pune relacije baze inicijalnim podacima.
		
			\begin{figure}[H]
				\includegraphics[scale=0.54]{Slike/Class Diagram5.PNG}
				\centering
				\caption{Razredi DAO sučelja}
				\label{fig:dijagram}
			\end{figure}
		
		Razredi DAO sučelja oblikuju sučelje za perzistenciju podataka. Također, tu je i konkretna realizacija DAO sučelja koja podatke sprema i čita u relacijsku bazu podataka - SQLDAO.	\newline \newline
		
		Servleti služe za obradu HTTP zahtjeva. Pomoću nekoliko slika koje slijede opisani su servleti za sve tipove korisnika te oni kojima je oblikovan REST API za spajanje Android aplikacije.
		
			\begin{figure}[H]
				\includegraphics[scale=0.5]{Slike/Class Diagram6.PNG}
				\centering
				\caption{Razredi Servleti - općenite operacije dostupne svim korisnicima}
				\label{fig:dijagram}
			\end{figure}
		
			\begin{figure}[H]
				\includegraphics[scale=0.5]{Slike/Class Diagram7.PNG}
				\centering
				\caption{Razredi Servleti - funkcije RESTful API-a}
				\label{fig:dijagram}
			\end{figure}
		
			\begin{figure}[H]
				\includegraphics[scale=0.5]{Slike/Class Diagram8.PNG}
				\centering
				\caption{Razredi Servleti - klijent}
				\label{fig:dijagram}
			\end{figure}
		
			\begin{figure}[H]
				\includegraphics[scale=0.5]{Slike/Class Diagram9.PNG}
				\centering
				\caption{Razredi Servleti - administrator}
				\label{fig:dijagram}
			\end{figure}
		
			\begin{figure}[H]
				\includegraphics[scale=0.5]{Slike/Class Diagram10.PNG}
				\centering
				\caption{Razredi Servleti - bankar}
				\label{fig:dijagram}
			\end{figure}
		
			\begin{figure}[H]
				\includegraphics[scale=0.5]{Slike/Class Diagram11.PNG}
				\centering
				\caption{Razredi Servleti - službenik za odobravanje kreditnih zahtjeva}
				\label{fig:dijagram}
			\end{figure}
		
		\section{Dijagram stanja}
			
			
			
			
			Dijagram stanja spada pod ponašajne i dinamičke dijagrame te prikazuje stanja objekta i prijelaze iz jednog stanja u drugo temeljene na događajima. Na slici 4.9 prikazan je dijagram stanja za registriranog korisnika. Nakon prijave, klijentu se prikazuje početna stranica na kojoj može pregledati svoj profil odnosno osobne podatke. Klikom na "Promijena lozinke" klijent mijenja svoju lozinku. Odabirom "Računa" klijent ima uvid u stanje svojih računa. Klikom na "Kartice" prikazuju mu se podatci o karticama te odabirom "Dodaj karticu" klijent može zatražiti novu kreditnu karticu koju odobrava službenik. Klijent svoje kredite može pregledavati klikom na "Krediti", a odabirom "Novi kredit" poslati zahtjev za novim kreditom službeniku za odobravanje kredita. Klijent odabirom "Otplati kredit" može prijevremeno otplatiti kredit. Pregled transakcija dostupan je klikom na "Transakcije", a klijent može obaviti prijenos sredstava s računa odabirom "Nova transakcija".  
			
			\begin{figure}[H]
				\includegraphics[scale=0.65]{Slike/dijagramStanja.PNG}
				\centering
				\caption{Dijagram stanja}
				\label{fig:dijagram}
			\end{figure}
			
			
			
			\eject
			
		
		\section{Dijagram aktivnosti}
			
		Dijagram aktivnosti je ponašajni dijagram i dinamički dijagram.Dijagrami aktivnosti vrlo su slični dijagramima stanja, ali s drugačijom
		semantikom elemenata s kojima se modelira. Primjenjuje se za opis modela toka upravljanja ili toka podataka. U modeliranju toka upravljanja svaki novi korak poduzima se nakon završenog prethodnog. Na dijagramu aktivnosti 4.10 prikazan je proces ugovaranja kredita. Korisnik se prijavi u sustav, klikne na "Krediti", i upisuje OIB klijenta koji je zatražio kredit. Korisniku se ispisuju krediti klijenta te nakon toga odabirom "Novi kredit" te unosi odgovarajuće podatke u polja. Nakon što je ispunio ispravno sva polja za unos, korisnik odabire "Provedi zahtjev" te se zahtjev za kredit sprema u bazu podataka.
		
			\begin{figure}[H]
			\includegraphics[scale=0.8]{Slike/dijagramAktivnosti.PNG}
			\centering
			\caption{Dijagram aktivnosti}
			\label{fig:dijagram}
		\end{figure}
	
	\eject
	
		\section{Dijagram komponenti}
		
			Dijagram komponenti prikazuje najvažnije komponente aplikacije i njihov međusobni odnos. Na web poslužitelju nalazi se web aplikacija i baza podataka. Web aplikacija komunicira s bazom podataka preko SQL upita koji se ostvaruju uporabom klase SQLDAO koja implementira DAO sučelje. Pomoću web preglednika korisnik pristupa web aplikaciji. Web preglednik i aplikacija komuniciraju pomoću HTTP protokola. Android aplikacija komunikaciju s web poslužiteljem ostvaruje spajanjem na RESTful dio web poslužitelja.
		
			 \begin{figure}[H]
			 	\includegraphics[scale=0.65]{Slike/DijagramKomponenti.jpg}
			 	\centering
			 	\caption{Dijagram komponenti}
			 	\label{fig:dijagram}
			 \end{figure}