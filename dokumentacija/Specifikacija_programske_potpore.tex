\chapter{Specifikacija programske potpore}
		
	\section{Funkcionalni zahtjevi}
			
			
			\noindent \textbf{Dionici:}
			
			\begin{packed_enum}
				
				\item Banka
				\item Administrator sustava
				\item Službenici banke
				\begin{packed_enum}
					\item Bankari
					\item Službenici za odobravanje kredita građana
				\end{packed_enum}
				\item Klijenti banke
				
			\end{packed_enum}
			
			\noindent \textbf{Aktori i njihovi funkcionalni zahtjevi:}
			
			
			\begin{packed_enum}
				\item  \underbar{Administrator - inicijator}
				
				\begin{packed_enum}
					
					\item Upravljanje sustavom
					\item Održavanje sustava
					\item Kreiranje, uređivanje i brisanje profila djelatnika i klijenata banke
					\item Uvid u popis korisnika
					
				\end{packed_enum}
			
				\item  \underbar{Baza podataka - sudionik}
				
				\begin{packed_enum}
					
					\item Sadrži podatke o korisnicima sustava
					\item Sadrži podatke o računima i karticama klijenata
					\item Sadrži podatke o transakcijama i kreditima
					\item Sadrži podatke o poslovnicama
					
				\end{packed_enum}
				
				\item	\underbar{Bankar - inicijator}
				
				\begin{packed_enum}
					
					\item Izrada i brisanje korisničkih profila klijenata
					\item Ažuriranje i mijenjanje korisničkih profila klijenata
					\item Otvaranje i zatvaranje transakcijskih i štednih računa
					\item Ugovaranje i deaktivacija debitnih i kreditnih kartica
					\item Ugovaranje kredita
					\item Preuzimanje klijentskih zahtjeva za sastankom
					
				\end{packed_enum}
			
				\item	\underbar{Službenik za odobravanje kreditnih zahtjeva - sudionik}
				
				\begin{packed_enum}
					
					\item Odobrenje i blokiranje kreditnih zahtjeva
					\item Uvid u podatke o klijentima
					
				\end{packed_enum}
				
				\item	\underbar{Klijent banke - inicijator}
				
				\begin{packed_enum}
					
					\item Pregledavanje podataka o vlastitom profilu
					\item Pregledavanje podataka o transakcijama
					\item Prijenos sredstava između vlastitih računa
					\item Prijenos sredstava na račun drugih klijenata
					\item Zahtjev za sastankom s bankarom u određenoj poslovnici u određeno vrijeme
					
				\end{packed_enum}
				
				\item	\underbar{Neprijavljeni klijent - inicijator}
				
				\begin{packed_enum}
					
					\item Upisuje OIB i ključ dobiven od bankara za otvaranje usluge internet bankarstva
					
				\end{packed_enum}
			
			\end{packed_enum}
			
			\eject 
			
			
				
			\subsection{Obrasci uporabe}
				
				\textbf{\textit{dio 1. revizije}}
				
				\subsubsection{Opis obrazaca uporabe}
				
					\noindent \underbar{\textbf{UC1-Dodavanje novih korisnika }}
					\begin{packed_item}
						
						\item \textbf{Glavni sudionik: }Administrator
						\item  \textbf{Cilj:} Dodati korisnika
						\item  \textbf{Sudionici:} Baza podataka
						\item  \textbf{Preduvjet:} Korisnik se ne nalazi u bazi podataka
						\item  \textbf{Opis osnovnog tijeka:}
						
						\item[] \begin{packed_enum}
							
							\item Administrator unosi potrebne korisničke podatke
							\item Podaci se unose u bazu podataka
							\item Administrator sprema korisnika u bazu podataka
							
						\end{packed_enum}
						
						\item  \textbf{Opis mogućih odstupanja:}
						
						\item[] \begin{packed_item}
							
							\item[1.a] Unos neispravnih podataka
							\item[] \begin{packed_enum}
								
								\item Ispis odgovarajuće poruke upozorenja
								
								
							\end{packed_enum}
							
							
						\end{packed_item}
					\end{packed_item}
				
				\noindent \underbar{\textbf{UC2-Izmjena osobnih podataka korisnika  }}
				\begin{packed_item}
					
					\item \textbf{Glavni sudionik: }Administrator
					\item  \textbf{Cilj:} Urediti i izmijeniti osobne podatke korisnika
					\item  \textbf{Sudionici:} Baza podataka
					\item  \textbf{Preduvjet:} Korisnik je prijavljen u sustav kao administrator
					\item  \textbf{Opis osnovnog tijeka:}
					
					\item[] \begin{packed_enum}
						
						\item Administrator pronalazi željenog korisnika
						\item Administrator odabire opciju uređivanja korisnika
						\item Administrator izmijenjuje podatke korisnika i sprema ih u bazu \\ podataka 
						
					\end{packed_enum}
					
					
				\end{packed_item}
				
				
				

					\noindent \underbar{\textbf{UC$<$broj obrasca$>$ -$<$ime obrasca$>$}}
					\begin{packed_item}
	
						\item \textbf{Glavni sudionik: }$<$sudionik$>$
						\item  \textbf{Cilj:} $<$cilj$>$
						\item  \textbf{Sudionici:} $<$sudionici$>$
						\item  \textbf{Preduvjet:} $<$preduvjet$>$
						\item  \textbf{Opis osnovnog tijeka:}
						
						\item[] \begin{packed_enum}
	
							\item $<$opis korak jedan$>$
							\item $<$opis korak dva$>$
							\item $<$opis korak tri$>$
							\item $<$opis korak četiri$>$
							\item $<$opis korak pet$>$
						\end{packed_enum}
						
						\item  \textbf{Opis mogućih odstupanja:}
						
						\item[] \begin{packed_item}
	
							\item[2.a] $<$opis mogućeg scenarija odstupanja u koraku 2$>$
							\item[] \begin{packed_enum}
								
								\item $<$opis rješenja mogućeg scenarija korak 1$>$
								\item $<$opis rješenja mogućeg scenarija korak 2$>$
								
							\end{packed_enum}
							\item[2.b] $<$opis mogućeg scenarija odstupanja u koraku 2$>$
							\item[3.a] $<$opis mogućeg scenarija odstupanja  u koraku 3$>$
							
						\end{packed_item}
					\end{packed_item}
				
					
				\subsubsection{Dijagrami obrazaca uporabe}
					
					\textit{Prikazati odnos aktora i obrazaca uporabe odgovarajućim UML dijagramom. Nije nužno nacrtati sve na jednom dijagramu. Modelirati po razinama apstrakcije i skupovima srodnih funkcionalnosti.}
				\eject		
				
			\subsection{Sekvencijski dijagrami}
				
				\textbf{\textit{dio 1. revizije}}\\
				
				\textit{Nacrtati sekvencijske dijagrame koji modeliraju najvažnije dijelove sustava (max. 4 dijagrama). Ukoliko postoji nedoumica oko odabira, razjasniti s asistentom. Uz svaki dijagram napisati detaljni opis dijagrama.}
				\eject
	
		\section{Ostali zahtjevi}
		
			\textbf{\textit{dio 1. revizije}}\\
		 
			 \textit{Nefunkcionalni zahtjevi i zahtjevi domene primjene dopunjuju funkcionalne zahtjeve. Oni opisuju \textbf{kako se sustav treba ponašati} i koja \textbf{ograničenja} treba poštivati (performanse, korisničko iskustvo, pouzdanost, standardi kvalitete, sigurnost...). Primjeri takvih zahtjeva u Vašem projektu mogu biti: podržani jezici korisničkog sučelja, vrijeme odziva, najveći mogući podržani broj korisnika, podržane web/mobilne platforme, razina zaštite (protokoli komunikacije, kriptiranje...)... Svaki takav zahtjev potrebno je navesti u jednoj ili dvije rečenice.}
			 
			 
			 
	