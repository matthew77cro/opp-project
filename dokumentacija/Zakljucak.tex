\chapter{Zaključak i budući rad}
		
		\textbf{\textit{dio 2. revizije}}\\
		
		 \textit{U ovom poglavlju potrebno je napisati osvrt na vrijeme izrade projektnog zadatka, koji su tehnički izazovi prepoznati, jesu li riješeni ili kako bi mogli biti riješeni, koja su znanja stečena pri izradi projekta, koja bi znanja bila posebno potrebna za brže i kvalitetnije ostvarenje projekta i koje bi bile perspektive za nastavak rada u projektnoj grupi.}
		 
		 Zadatak našeg tima bio je razvoj web i mobilne aplikacije za online bankarstvo. Realizacija projekta podijeljena je u dvije faze.
		 
		 U prvoj fazi projekta okupljen je tim za razvoj aplikacije. Na prvom sastanku definirana je ideja projekta te izvršena podjela uloga članovima tima. 
		 U prvoj fazi definiran je i napravljen temelj cijelog sustava. Značajan dio dokumentacije napisan u ovoj fazi projekta uvelike je olakšao daljnji rad pri realizaciji aplikacije. Pisanjem funkcionalnih i nefunkcionalnih zahtjeva, crtanjem uml dijagrama, izradom modela baze podataka pomoglo se članovima tima zaduženim za implementaciju aplikacije.
		 
		 U drugoj fazi projekta naglasak se stavlja na implementaciju aplikacije. Članovi tima zaduženi za backend i frontend intenzivno rade na realizaciji aplikacije kako bi uspjeli implementirati sve dogovorene funkcionalnosti aplikacije. Ostatak tima bio je zadužen za crtanje preostalih uml dijagrama, testiranje i dovršavanje dokumentacije. Zahvaljujući dobro definiranim temeljima projekta u prvoj fazi dokumentacije, ubrzali smo fazu implementacije te spriječili moguće pogreške. Članovi tima stekli su znanja o raznim tehnologijama i alatima za koje su pokazali interes i s kojima će se možda susresti u budućnosti.
		 
		 Nakon gotove aplikacije i dalje postoji prostor za usavršavanje. U budućnosti bilo bi potrebno doraditi mobilnu aplikaciju kako bi sadržavala sve funkcionalnosti kao i web aplikacija.
		 
		 Zadovoljni smo postignutim rezultatom te stječenim znanjima kroz ovaj projekt. Osim susretanja sa novim tehnologijama i sudjelovanju u procesu implementacije aplikacije, stekli smo osjećaj rada u timu te važnosti dobre organizacije i pravilne raspodjele poslova.
		 
		  
		 \textit{Potrebno je točno popisati funkcionalnosti koje nisu implementirane u ostvarenoj aplikaciji..}
		
		\eject 